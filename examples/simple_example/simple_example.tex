%Kompiluj z lualatex
\documentclass[magister,druk,polski]{dyplompwr}
\usepackage{fontspec}
\usepackage[utf8]{luainputenc}
\usepackage{polyglossia}
\setdefaultlanguage{polish}
\usepackage{lipsum}
%==================
%   Mikrotypografia
%==================
%Super pakiet, który poprawia ułożenie liter i w efekcie wygląd dokumentu
\usepackage{microtype}

%Tutaj już są metadane potrzebne do szablonu dyplompwr
\author{Imię i nazwisko autora}
\title{Tytuł  polski}
\titlen{English title}
\promotor{dr hab. inż. Pan Promotor}
\wydzial{Wydział Chemiczny}
\miejscowosc{Wrocław}
\kluczowe{słowa kluczowe: jedno, drugie, trzecie}
\streszczenie{Kilka słów o pracy zwane też streszczeniem}
\dedykacja{dedykacja}

\begin{document}    %Dopiero tutaj zaczyna się zawartość dokumentu

\maketitle          %Tworzenie strony tytułowej
\tableofcontents    %Tworzenie spisu treści
\listoffigures      %Tworzenie spisu figur
\listoftables       %Tworzenie spisu tabel

\chapter{Krótki przykład użycia pakietu dyplompwr}
\lipsum
\end{document}
