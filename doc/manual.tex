\documentclass[12pt,a4paper]{article}
\usepackage[utf8]{luainputenc}
\usepackage{lmodern}
\usepackage{polyglossia}
\setdefaultlanguage{english}
\usepackage{verbatim}
\usepackage{url}
\usepackage[colorlinks=true]{hyperref}
\usepackage{cleveref}
%\usepackage{cleveref}
\author{Robert Kubosz \\ \href{mailto:kubosz.robert@gmail.com}{kubosz.robert@gmail.com}}
\title{dyplompwr}
\begin{document}
\maketitle
\section{Summary of package}
\label{sec:wstep}
\par This package delivers a~template of thesis for students of Wrocław
University of Technology. This package is based on other thesis templates
introduced by dr Wojciech
Myszka\footnote{\url{https://kmim.wm.pwr.edu.pl/myszka/projekty/klasa-do-skladu-pracy-dyplomowej-magisterskiej-i-inzynierskiej-na-wydziale-mechanicznym-politechniki-wroclawskiej/}}
and dr Andrzej Giniewicz\footnote{\url{https://github.com/aginiewicz/pwrmgr}}.
I have modified them to fulfill requirements of Faculty of
Chemistry\footnote{\url{http://www.wch.pwr.edu.pl/druki_dyplomanci,11.dhtml}}.
\par This package creates title page and provides proper document format:
\begin{itemize}
    \item archive version has margins 2,5 cm top and bottom, 2 cm outer and 3,5
        cm inner. There is generated a pdf for two-sided printing with leading
        1.
    \item version for supervisor is compiled with margins 2,5 cm top, bottom
        and right, left margin is 3,5 cm wide. Leading is equal to 1,5.
        Generated pdf is for one-side printing.
\end{itemize}
This package can generate a title page for master thesis or engineer's thesis.
\section{Requirements}
\par In order to generate a proper title page two additional fonts (URW
Garamond and URW Classico)  are
required\footnote{\url{http://pwr.edu.pl/uczelnia/o-politechnice/materialy-promocyjne/logotyp}}.
\par To install these fonts visit page linked in footnote and follow
instructions\footnote{\url{https://www.tug.org/fonts/getnonfreefonts/}}. Users
of Arch Linux, who install dyplompwr from aur can automatically resolve font
dependencies.

\section{Installation}
Package can be installed manually in home directory:
\begin{verbatim}
    $HOME/texmf/tex/latex
\end{verbatim}
or just copy package files into thesis directory.

Users of Arch Linux can install the package from
AUR\footnote{\url{https://aur.archlinux.org/packages/dyplompwr/}}.

\section{Usage}
\par Packet should be loaded in first line of~.tex file:
\begin{verbatim}
\documentclass[package options]{dyplompwr}
\end{verbatim}
where \verb|package options| can be replaced with student's preferences:
\begin{itemize}
    \item thesis type: \verb|master| lub \verb|engineer| will generate
        headers for master's or engineer's thesis accordingly;
    \item formatting: \verb|archive| generates version for archive,
        \verb|oneside| generates version for supervisor (for more info see
        \cref{sec:wstep});
    \item thesis language: \verb|pl| for polish thesis, \verb|en| for english thesis.
\end{itemize}
If no options are provided, then there is generated by default a polish master thesis with formatting for archive.
These options can be also provided by hand:
\begin{verbatim}
    \documentclass[master,archive,pl]{dyplompwr}
\end{verbatim}.
To get an english engineer's thesis for supervisor first line of~.tex file
should be like below:
\begin{verbatim}
    \documentclass[engineer,oneside,en]{dyplompwr}
\end{verbatim}.
\subsection{Metadata}
Title page of thesis contains data such as polish and english title, author's
name, supervisor's name, faculty, keywords and abstract. To provide it there
should be filled metadata in document preamble as in example below:
\begin{verbatim}
    \documentclass[master,oneside,pl]{dyplompwr}
    \author{author's name}
    \title{polish title of thesis}
    \titlen{english title of thesis}
    \supervisor{name and titles of supervisor}
    \faculty{name of faculty}
    \city{Wrocław or other city}
    \keywords{phrases that describes thesis}
    \abstract{short summary of thesis}
\end{verbatim}
Metadata should be provided before \verb|\begin{document}| in~.tex file.
\end{document}
